\documentclass[11pt]{article}
\usepackage{amsmath}

\newcommand{\hl}{\mathcal{H}}
\newcommand{\zl}{\mathcal{Z}}
\newcommand{\upp}{\uparrow}
\newcommand{\dnn}{\downarrow}
\newcommand{\bs}[1]{\boldsymbol{#1}}

\begin{document}

%%%%%%%%%% 
\section{Introduction}
For text-segmentation, a non-deterministic search by mean of {\em
  Simulated Annealing} {\bf(SA)} has been used. In the oroginal
implementation the interaction between segments has been taken into
account by scanning through the text and evaluate the segmentation
based on two variables
%% 
\begin{itemize}
\item lexicon length
\item segment length
\end{itemize}
%%
That can be reformulated in terms of repetition of a segment and its
length.

Here we propose the Markov transition matric as an
interaction between segments to evaluate the free energy more
accurately.

\begin{itemize}
\item Overview of simulated annnealing
\item Search for the ground state
\item Ising model and its relevance to text segmentation
\item Interactive and non-Interactive system
\item Spin lakes and Ferromagnetism 
\end{itemize}

%%%%%%%%%% 
\section{Statistical Mechanics (Review)}

Lets consider a spin chain of $N$ sites in presence of an applied
magnetic filed. The Hamiltonian (energy) of the system is
%%
\begin{equation}
  \hl = \bs{B}\cdot \bs{\mu} = \mu_0 B_0 \sum_i s_i
\end{equation}
%%
Let $n_\upp$ and $n_\dnn$ represent the number of spins in
the chain that are parallel and anti-parallel to $\bs{B}$,
respectively. To follow {\em Macrocanonical} formalism, we write down
the total number of accessible energy state
%%
\begin{equation}
  \Omega = \frac{N!}{n_{\upp}! n_{\dnn}!}
\end{equation}
%%
and we know
%%
\begin{equation}
  n_\upp + n_\dnn = N, \qquad n_\upp - n_\dnn = \frac{E}{B_0 \mu_0}
\end{equation}
%%


%%%%%
\subsection{Canonical Approach}
The partition fucntion reads
%%
\begin{equation}
  \zl = \sum_i e^{\beta \epsilon_i}
\end{equation}
%%
where for a spin-chain in presense of applied mangetic field is
%%
\begin{equation}
  \zl = \sum_{n_\upp} e^{\beta (n_\upp - n_\dnn)\mu_0 B_0}
\end{equation}
%% 
%%%%%%%%%%
\section{Spin Colony}
Partialy ordered system
\end{document}
